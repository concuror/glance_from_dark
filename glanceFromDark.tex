
\documentclass[a4paper,oneside]{book}
\usepackage[no-math]{fontspec}
\usepackage{polyglossia}
\usepackage{epigraph}

\setmainlanguage{ukrainian}

%\setmainfont[Mapping=tex-text]{Linux Libertine O}
\newfontfamily\ukrainianfont[Script=Cyrillic]{Linux Libertine O}

\begin{document}

\title{Погляд з пітьми}
\author{Тітов Андрій}
\date{December 13 2007}
\maketitle

\tableofcontents

\part{Аналіз світла}
\setcounter{chapter}{1}

\epigraph{And what do you see int the inc?}{Dr. Killjoy}

\section*{}

Заряд електричного струму, котрий віддавав блювотно-фіолетовим відтінком чи то
смаком пробігся по старому дев'ятнадцяти-міліметровому проекторові, змусивши
того похитнутися на своїй нестійкій конструкції. Промінь чистого світла
вирвався зі старого апарату і спроектувався  на обшарпану шпалерку. Блискучий
сіруватий скальпель плавно почав врізатися у черепну коробку піднімаючи
фонтани крові. Різко завила циркулярка вгризаючись у кістки. Через двадцять
хвилин ослизла сіра речовина тихо булькнула в спиртовому розчині. Тонкий
промінь світла швидко забігав по старому пожовклому журналі з описами
експериментів, де рівним почерком виводились букви: “Препарація пацієнта
завершена успішно. Великим півкулям головного мозку пошкоджень не
нанесено. Психологічні травми спричинені невідповідністю фізичного носія
свідомості почали зникати. Повне одужання очікується у найближчі десять
тижнів. Фізичний носій (тіло пацієнта) повністю ліквідовано. Лікування
перейшло до останньої стадії. Статус пацієнта: задовільний.”.

Розряд яскраво-жовтої блискавки на секунду яскраво висвітив старий будиночок
вікторіанського стилю посеред великого, акуратно доглянутого саду, увінчаного
зеленим лабіринтом. Одинокий наглядач знову акуратно закутався у свій плащ
задля захисту від зливи та вітру. Він бачив яскравий блиск золотої таблички на
покинутому маєтку, як бачив кожен наглядач, котрий принаймі раз патрулював
територію в грозу. Але ніхто ніколи навіть не доторкнувся до неї – надто добре
засіла в пам'яті та дивна історія з Браяном (тим придурком, котрий явно не
вмів управлятися зі своїм дебелим тілом).

На небосхилі знову з’явилася яскрава іскра, а у відповідь їй на стіні маєтку
блиснула отруйно-жовта вивіска. Кожен охочий міг прочитати на ній величними
буквами висічений надпис: “Карнейтський заклад для легко помішаних та буйно
хворих імені доктора Кілджоя”. Але охочих не знайшлося...

\section*{} 
 
Перегорнувши останню сторінку, рука потягнулася за чашкою кави. “Ах, Шекспір
навіває на такі драматичні і глибоко-філософські роздуми. Шкода, що мої
пацієнти надто зациклені на самих собі, щоб поговорити про це. Але вони дають
плідну ниву для вивчення людського мозку. Сотні відхилень від “норми”, і всі
вони найгостріше проявляються саме тут, на цьому острові. Цікава
закономірність, правда, друже?”.

Особа, до якої звертався автор монолога, лежала скулившись у темному закутку і
маніакально похрипуючи, вилизувала руку. “Ось наприклад Ви. Ви думаєте і
ведете себе наче собака. Причому цей психоз почав розвиватися виключно тут на
острові. Проте хіба це погано? В Балтіморі Ви вбили понад п’ят\-над\-цять
людей, в тому числі дітей та жінок. А зараз ви стали вірним і відданим
помічником. Отже постає логічне питання: чи вартує лікувати Ваш мозок? З
одного боку цього вимагає моя професія з її клятвою, а з іншого окрім веселої
компанії Ви становите собою цікавий експеримент! Але не час для сумних
роздумів, робота чекає. Є одна річ котра справді потребує моєї цілковитої
уваги. Адже що може бути важливішим за підготовку до приходу мого
найскладнішого пацієнта. Сотня інших не варта такої уваги, якої вартий цей
індивідум. Я відчуваю, що найважливіша загадка цього острова вирішиться лише з
його приходом. Як сумно, що я не побачу цього своїми власними очима. Саме тому
я маю закінчити цей величний проект ще до того, як останній імпульс покине мій
мозок і я на завжди втрачу можливість розгадати найпрекраснішу загадку в
історії.”.

\section*{}
 
Стояла пізня осінь, і старий клен із неприємним скрипом стукав голою гілкою по
вікну. Похмура, дощова погода здавалося поселилася у цьому мальовничому
куточку острова разом із новим мешканцем старого будинку – пережитку
вікторіанської епохи. Місцева влада вже покинула останні надії продати цю
землю і кинула будівлю на призволяще, коли прийшов цей дивний чоловік зі
східного Балтімора. І хоча за ним тягнулася темна  і зловіща історія про
зникнення батьків (дехто навіть подейкував, що це сам хлопчик і вбив їх),
проте його документи були в порядку і губернатор без вагань оголосив старий
маєток проданим задля перетворення його на приватну експериментальну
психіатричну лікарню. Її управляючим і єдиним працівником проголосили доктора
медицинських наук Кілджоя.

Десять років пройшло з того часу і лікарня перебрала у доктора його темну
славу. Жодна здорова людина не сміла переступати межу мальовничого саду даного
закладу. Проте зараз кожен ненормальний в своїй камері дивився у власне
маленьке віконечко, намагаючись розгледіти дивну маленьку дівчинку вдягнену в
старомодний пуританський одяг, що пробиралася через лабіринт, залишаючи
доріжку із обпаленої трави. В руках вона несла старенького обшарпаного
ведмедика. Занепокоївшись через дивну для цього часу доби тишу, Кілджой
відклав у сторону старий пожовклий рукопис, де у неясному мерехтінні полум’я
чітко виднівся лише один уривок фрази: “Ти менший за ... але і
більший!”. “Добрий” доктор підійшов до вікна і здивовано підняв брови. На
веранді стояла маленька дівчинка із зачіскою, що тонким мереживом чорного
волосся прикривала обличчя. В лівій руці міцно затиснене ведмежа, а правою
дитя зосереджено водило по стіні старого маєтку. Доктор здивувався не через
старомодний вигляд гості, і не через її появу (це звичайно не було зовсім
нормальним, але скоріш за все мало просте пояснення). Що було справді дивним,
так це маленький димок, що здавалось виходив із її пальця. Не надто довго
роздумуючи хазяїн попрямував до дверей прямо у своєму докторському
халаті. Проте коли двері безшумно розкрилися веранда виявилася порожньою. На
стіні, котра пахла розпеченим вапном (дивно, адже та була базальтовою!)
виднівся пульсуючий блідо-червоний надпис: “Дочекайся його, він
прийде”. Раптом в голові доктора прозвучав дитячий голос. Позбавлений дорослих
емоцій і сповнений дитячої грайливості, він говорив такі важкі речі, що під
шкіру заповзав могильний холод. “ Знайди вихід... Він прийде із демонами в
собі і навколо себе... поможи йому, поможи нам... Звернись до науки,
відвернись від зла... відкрий очі ув’язненому!”. Заходячи до вітальні доктор
твердив про себе , що усе має логічне пояснення. Того вечора помер один із
найбільш багатообіцяючих експериментів...

\section*{}
 
“Шкода, що ти не дожив до цього, друже. Принаймні ти не мучився, я чув що
Гермес справжній професіонал. У нього велике майбутнє в такому місці як
Аббот.” Голос здавалося ішов звідусіль. Кімната була сильно зашторена і жодний
промінь сонця (котре усе рівно було за хмарами) не проникав всередину. Проте
жовтувате світло наповняло великий конус що розповсюджував старенький
кінопроектор. “Остання фаза переходу, свідком якого ти був, практично
завершена. Я вже можу передавати свій голос і взаємодіяти із навколишнім
середовищем. Але саме сьогодні все завершиться: за допомогою хімікатів
власного виготовлення мені нарешті вдалося модифікувати кіноплівку достатньо,
щоб передавати моє зображення із цього місця. Місце до речі жахливе, але
будучи не пов’язаним  із часом, воно ідеально підходить для моїх цілей. Так я
не лише зможу дочекатися потрібного моменту, але й продовжити свої
дослідження. Таким чином апокаліпсис не застане мене непідготовленим.”

Страшний гуркіт потряс столітню будову лікарні до самого фундаменту. Маленький
моток червоної мов кров плівки перемістився до механізму проектора. Світло
замигтіло, і посеред кімнати з’явився силует високого статного мужчини у
докторському вбранні, оббризканому кров’ю біля подолу. Чоловік оглянув себе,
доторкнувся руками, узяв старе вишукане дзеркало і довго оглядав себе. “Ах, як
приємно знову тут опинитися. Отже лікарня знову відкрита і очікує
пацієнтів!”. Маніакальний сміх на довгі секунди наповнив спустошений комплекс,
знаходячи кожну камеру і кожен коридор запустілого будинку.

\section*{}
 
Крупний штатив задів стару пурпурову штору і до кімнати увірвався порив
вологого вітру. Новенький лабораторний журнал відкрився на другій сторінці. “Я
довго розмірковував про способи відтворення особистості у просторі.” – було
акуратно виведено бездоганним аристократичним почерком: “Магія і спіритизм
були відкинуті першими як абсолютно невірогідні та абсурдні. Довгі ночі я
сидів за давніми перекладами Біблії та Корану, але і тут мене спіткала невдача
– не зважаючи на обширну інформацію про збереження свідомості у вічності,
жоден із них не перетинався із нашою площиною буття. Саме тоді я зрозумів що
єдиний справжній вихід зможе дати лише наука. Після обрахувань надто складних,
щоб передати їх паперові, я прийшов до висновку, що сфокусований потік світла
може стати прекрасним провідником між вимірами. Отже об’єктами, що можуть мені
допомогти є або радіоактивні елементи, або прожектора або кінопроектори. Перші
надзвичайно рідкісні, а другі – банальні. І тут як завжди нас виручать
мистецтво і наука. Наука кіномистецтва.”

\part{Живучий смертю}
\setcounter{chapter}{2}

\epigraph{Life is cruel: first it gives us, then it takes it away}{Hermes T. Height}

\section*{}

Відпрацьованим рухом рука натиснула на кнопку. Іще один день на роботі – іще
одна паскуда засуджена до смерті отримала по заслугам. Зелений газ із
неприємним запахом метанових сполук повільно заповнив газову
камеру. Бритоголовий в’язень у стандартній помаранчевій формі відчувши себе у
небезпеці підняв голову із виколотими очима і почав хапати ротом повітря. Руки
панічно застукали по армованому склу даремно намагаючись розбити його. Життя
бідолахи повільно поглиналося бридкими отруйними випарами. З глухим стуком
тіло впало на залізну підлогу. “Ах, хто сказавcc, що робота закінчується разом
із життямм? Моя робота кінця не знає!”. Рука з тихим шипінням зникла із
кнопки, і вслід за цим зеленувата хмара, що нагадувала газ у камері, піднялася
крізь вентиляційну решітку контрольного пульта. 

Через секунду в газовій камері з’явилися два охоронця. “Тут іще один. І на
плівці як завжди все пусто!”. “Знаєш, мені їх не шкода, але я боюсь за нашу
роботу” - прозвучала відповідь. Раптом газ знову заповнив камеру, і на підлогу
впало два нових трупа. “Цікаво, вони навчатьссся чи ні? Я такcc люблю свойю
роботу...”.

\section*{} 

“Перший раз?” – спитав офіцер у молодого охоронця, котрому доручили вперше
привести смертний вирок у виконання. Ця робота рахувалася тут в Абботі брудною
і недостойною офіцера або вислуженого охоронця. На цю роботу присилали лише
зелених новачків або попавши в немилість начальства свого блоку. Навіть
найбільші садисти не могли довго витримати тут, посилаючи незнайомих людей на
той світ по кілька на день.“Дивись не обблюй всю підлогу!”. Не отримавши
відповіді офіцер насупився і відвернувся до дверей, через які саме заводили
нового смертника.

Як і переважна більшість із них, цей засуджений був давно зламаним і не
пробував чинити жодного опору і навіть не протестував. Цей суб’єкт давно
змирився із своєю долею. Його швидко посадили на крісло і приготували до
страти. Новенький опустив важіль і по дротам тихенько задзижчав
струм. Смертник здригнувся раз, другий; тихо захрипів і підняв очі на
охоронців, що стояли в залі. В очах стояли сльози і завмирали іскри страшної
муки. Хрип припинився, а з ним і дзижчання струму. В’язень так і помер, з
виразом печалі на обличчі. Незворушним стояв лише новенький, спокійно
оглядаючи труп і міркуючи про щось своє. Офіцер відразливо гмикнув собі під
носа і швидким кроком вийшов з приміщення. За ним прослідувала решта
охоронців, лишивши винуватця їхнього поганого настрою одиноко оглядати справу
рук своїх. “Думаю, мені сподобається ця робота” задумливо сказав новенький.

\section*{}
 
Це місце виявило його справжню натуру. Воно вміло виявляти справжніх хижаків,
справжніх вбивць. Йому пощастило, робота служила гарним прикриттям. Сама
в’язниця здавалось допомагала своєму новому катові, і за це останній вірно
служив їй, акуратно і точно виконуючи свою роботу. Адже він не міг не відчути
яку владу і силу давала йому ця посада саме тут, в Абботі. До нього приводили
людину, котру він бачив вперше. Ця людина нічим не завинила перед ним і ніколи
взагалі не пересікала дороги свого вбивці, але він був відповідальним за самий
важливий момент життя засудженого. Це справжня сила!

За ті кілька років, котрі пройшли за роботою він перепробував усе: електричний
стілець, голка з наркотиками. Але улюбленим для нього залишався газ. Жертва не
зазнавала зайвих ушкоджень і до самої смерті розуміла, що з нею
відбувається. Саме він придумав додати до безбарвного і непомітного газу
домішки, що перетворили його на зеленого смердючого вбивцю, що повільно
пробирався до організму жертви і висмоктувала її життя. Деколи здавалося, що
саме це і відбувається, і еманації життєвої сили приреченого просочувалися
крізь стіни і наповнювали Аббот енергією. В принципі так воно і є...

\section*{}

Кран повернувся, і маленький струмінь газу зашипів, вириваючись під
тиском. Велика хмара бридкої на вигляд зеленої субстанції зависла в
коридорі. У центрі аномалії проглядувала примарна фігура у формі із темною
відзнакою тюремного виконавця смертних вироків. В’язниця була за непоєна і цей
неспокій передавався її слугам. Дух відчував приховану загрозу їхнім планам і
тому поспішав знайти причину занепокоєння.

Ніщо не мало завадити приходу обраного, котрий розширить їхню владу не лише на
цілий острів, але й на континент. Саме тоді він зможе по справжньому
вирішувати кому жити, і хто для цього за слабкий. Він створить ціле
суспільство вбивць, де правити буде одноособово і нескінченно. Але для цього
необхідно перетворити Обраного на одного із них, навчити його виконувати
брудну роботу машини смерті, перетворити його внутрішніх демонів на демонів
зовнішніх і випустити цю хвилю терору на вільний світ жалюгідних слабких
смертних з метою тотальної зачистки. Проте зараз відчувалася загроза саме цій
частині плану. Аббот завмерла прислухаючись. Вона навіть покинула свою нову
іграшку, задля визначення загрози глобальним планам.

Острів вже не раз показував своє невдоволення своє вільністю цієї порівняно
молодої, але сильної і гордої споруди. Привид побоювався, що саме острів знову
намагається сплутати їхні карти. Він вже створив перепону, заінтригувавши
цього мерзенного доктора. І тепер очевидно Карнейт шукає агентів тут всередині
самої в’язниці.

Розуміння прийшло раптово, разом із вимкненням світла. “ А це мисс ще
подивимосся.” – неприємним сиплим голосом прошипіла фігура в хмарі. Настрій
цієї примари миттєво змінився, і по коридору пройшовся порив затхлого повітря,
котрий приніс бридкий сморід гнилої плоті  у суміші із запахами тухлих яєць,
холодного поту та еманаціями смерті і панічного страху. Вслід за цим продихом
стрімко понісся і потік газу, колір котрого помінявся із блідо-зеленкуватого
на насичений зелений колір пронизаний злобою так, що її здавалося можна
розгледіти.

Хмара зупинилася у великому холі і потягнулася щупальцями, що нагадували
покриті слизом і тином щупальця  молюска, у всі кутки приміщення, наповнюючи
його зловіщими зеленими стовпами диму. Через кілька секунд двері відчинилися і
до зали увійшла дівчинка, яку явно не очікуєш побачити в такому місці. На ній
було вицвівши плаття старого покрою. Дитина прийшла босоніж і стискаючи
ведмедика у лівій руці. Разом із загадковою дитиною до приміщення зайшла і
хвиля палючого жару, що здула ближні прядки мерзотного туману. “Ах Гермес!
Вірний пес захищає свою господиню. Як зворушливо...” прозвучали дивовижно
сильні і чисті слова, кожне з яких заставляло зелену примару здригатися мов
від удару батога. “А тепер відійди в сторону або ми зметемо тебе, як скоро
зметемо і це прокляте усіма смертними і безсмертними місце”.

Дух у формі коротко розсміявся придушеним маніакальним сміхом, від котрого
будь-кого охопив би не лише жах, а й непереборне бажання опинитися в туалеті,
вивертаючи рештки свого обіду. Махнувши рукою, примара направила усі огидні
болотного кольору відростки хмари на нахабне дівчисько. Проте щупальця
розсіювалися не доповзаючи до своєї цілі. В повітрі запахло чимось потворним,
що нагадувало суміш запахів горілої плоті і гниючих трупів.

Так, кат Абботу був справжнім майстром смертних вироків і приводив їх у
виконання не дивлячись на супротив об’єктів. Але зараз йому протистояло дещо
більше за нього, навіть враховуючи силу надану збоченому духові
в’язницею. Острів гріхів з давніх-давен тримав свої таємниці, вирощуючи
ідеальні інструменти своєї волі. А зараз діяв один із найвідточеніших
зразків. Сила дівчинки неначе вогненним смерчем вдарила по рабу
в’язниці. Чорний дим затягнув фігуру в центрі газової хмари. Струмені
смертельного отруйного повітря на кинулись на  дитину з ведмежам і все
перемішалося у битві двох спотворений розумів. Через  кілька хвилин дівчина у
вестибюлі залишилась сама, і лише зелений димок навколо вентиляційних виходів
нагадував про те що відбулося. “Я думаю ви не розумієте того, щосс тут
дієтьссся. Нічого, сскоро ми зустрінемося і ти побачиш...”. Не зважаючи на
останню фразу, дівчинка похитуючись і залишаючи на підлозі оплавлені сліди
пішла далі, уважно вглядаючись у електропроводку, і кожні кілька метрів
вибухаючи. Як це не дивно, але охорона в залі спостереження не помітила
нічого: відеокамери просто не зафіксували драматичного двобою. Зате стіни
тепер уважно стежили за непроханою гостею і здавалось, намагалися роздавити
останню...

\section*{}
 
Вчора я отруїв в газовій камері іще одну людину. Робота більше не приносить
мені задоволення. З часів, відколи я підсмажив того виродка, котрий прямо тут
забив насмерть свою дружину, це почало перетворюватися на статистику. Навіть
газова камера не втішає мене. Але ці стіни, я відчуваю як вони нашіптують мені
вихід, котрий надасть мені іще біль влади над смертю. Можливо настав час
зробити цей останній крок і прийняти вищу відповідальність? Просунути свою
роботу на новий, непідвладний людям рівень? Так цій в’язниці не вистарчає
нового, вищого судді. Судді, котрий винесе вирок усім!

\section*{}
 
По всій в’язниці носилося відчуття незрозумілого очікування. Воно передавалося
всьому: рослинам, тваринам і навіть людям. Нетерпеливо намотували круги по
камерам в’язні. Наглядачі нервово покрикували на своїх підопічних. Нетерпіння
посилювалось за рахунок відсутності будь-яких показових страт чи покарань. І
серед усієї цієї метушні, що нагадувала відразливу метушню личинок в купі
гною, спокій діяла лише одна людина – виконувач смертних вироків на прізвисько
душитель. Він рутинно зайшов на місце своєї роботи і з непроникним виразом
обличчя натиснув на важіль. Потім вийшов у коридор, звернув у перші двері
направо і підійшов до великої конструкції газової камери, що саме наповнялася
дикою болотно-зеленою отруйною сушшю газів. Спрацював годинниковий механізм і
ручка повернулася впустив смердючу огидну речовину в зал. Кат глибоко вдихнув
газ, але на відміну від своїх жертв не закричав, не за панікував, а спокійно
віддав частинку своєї хворої сутності кам’яному вампіру – будівлі, котра за
довгі вки смертей навчилася живитися ними. Проте Аббот не зажер цю вільну
енергію. Насправді він зібрав кожен грам того, що колись було тюремним
виконавцем, і почав обережно вплітати в його структуру газ, забравший його
життя.

Через годину у віддаленому блоці “А” з вентиляції з шипінням  вилетіла хмара
отрути. Вона охопила двох необережних в’язнів, котрі практично миттєво
вдихнули смертельну дозу зараженого повітря. “Ах, я таксс люблю цю роботу. А в
смерті вона ще чарінішасс.” Другий етап плану по зустрічі обраного було
виконано, і стара будівля колоніального форту затряслася у приступах
надприродного сміху...

\part{Іскра надії}

\setcounter{chapter}{3}

\epigraph{Have you ever loved someone so much, that better kill her than see
  her with someone else?}{Horace Gauge}

\section*{}

В’язниця відлякувала навіть їх. Захоплена у темний кокон, пульсуюча
незрозумілими темними енергіями, що живили це місце впродовж усієї кривавої
історії. Проте лише  знедавна достатньо сильна, щоб змагатися із островом в
боротьбі за загублені душі. Вона добре приховувала свої таємниці, і на відміну
від навколишніх територій спрямовувала сили виключно на зло. Ця темна будівля
теж відчувала наближення обраного, але хотіла маніпулювати ним в своїх
інтересах. Однак жоден план не може бути надійнішим за інструменти, котрими
його виконують. А Аббот ще не вмів розбиратися в людях, і це було на руку
острову. Але не дивлячись на  всю некомпетентність, старі стіни випромінювали
майже видиму енергію образи, поневолення, злоби і надприродного відчаю. І
в’язниця готова була застосувати цю силу. Вони боялися. Вони боялися підвести
острів. Але іще більше вони боялися перемоги цієї злої споруди. Три блискучі
фігури із яскравим але безшумним вибухом з’єдналися, і на дорозі з’явилася
самотня маленька дівчинка. Залишаючи своїми босими ногами сліди на бетоні і
камені вона тихо закрокувала до в’язниці Карнейту...

\section*{} 
 
Вона не відпускала мене. Кожен день в один і той самий час я опинявся на
стільці і відчував, як струм проходить крізь кожен нерв мого тіла, викликаючи
нелюдські страждання. Незліченну кількість разів переживав я агонію останніх
хвилин мого життя. Не тішив мене навіть факт того, що кат ненабагато пережив
мене: говорили, що він надихався отруйного газу в сусідній кімнаті, покінчивши
із собою за допомогою улюбленої кари. Єдине, що втішало – це те що моя кохана
більше ніколи не буде самотньою і незахищеною. Я подбав про це. Добре
подбав. Пам’ятаю, як вона не хотіла приймати мої докази, пам’ятаю як ми
кричали один на одного, але вкінці кінців вона не мала іншого вибору. Тепер
вона ніколи не буде страждати, хоча і виглядала мила дещо неохайно, коли її,
із розбитою головою, піднімали з підлоги охоронці. Але так всім спокійніше, чи
я неправий?..

\section*{} 
 
Крик припинився, і в залі із електричним стільцем стало тихо. Проте здавалося,
що із усіх кінців блоку “Д” чується глухий безжально-наївний сміх сповнений
безглуздої радості і шаленства. Здавалося сміх линув із самих стін темної
в’язниці. Раптом голубий спалах пробігся по великим дротам високої напруги і
світло погасло не лише у блоці, а й в усій будівлі. Приглушений сміх змінився
подихом здивування і гніву. В’язниця здригнулася і замовкла, показово
“відвернувшись” від своєї забавки.

Через кілька хвилин нелюдський хрип вирвався із проводки разом із  голубою
дугою розряду. Обпечена фігура в дротах електропроводки вирвалася зі стіни
разом із значним куском останньої. Якби спалене обличчя не залишилося без очей
на них можна було б побачити здивування та надзвичайну підозру. Діялося щось
неправильне: повітря навколо було наповнене палким протистоянням. Складалося
враження, ніби хтось намагаючись залишитись непомітним, розкидував навколо
феєрверки. Відчувався запах чогось зловіщого, чогось палаючого ненавистю і
непорозумінням. Можна було практично осягнути запах горілої плоті, що
породжував блювотні рефлекси. Але зараз він відрізнявся від того, котрий
заставляв покоління охоронців зажимати рукою обличчя і бігти до найближчого
туалету  подавлюючи приступи прогіркло-кислої блювоти в горлі: у ньому не
вистачало свіжуватого запаху озону, котрий довго не вивітрювався після
використання “крісла”. Зараз це було вільне незалежне полум’я таке незвичне у
цих стінах вічності.

Раптом коридор накрила хвиля насиченого кислотного жару. Стіни вибухнули
безшумними вибухами. В коридорі з’явилась маленька дівчинка в пуританському
рожевому платті, із старим облізлим ведмежам в лівій руці. Гримаса злості і
відчаю явно читалася на її обличчі, незважаючи на довге волосся, що спадало їй
на очі, роблячи весь її образ дещо розгубленим і трагічним. “Нас
помітили... Немає часу... маєш відкритися... послухай насс...”. Неначе тавро,
розпеченим пульсуючим металом ці слова вгризалися в мозок заповнюючи його
біллю та відчаєм. Така сила не могла належати світу живих, і у той самий час
не була породженням в’язниці. Але звідки тоді взялася ця дівчинка? Чи варто
довіряти чомусь такому, що не зупинили навіть ці стіни? Але часу не
вистарчало...

“Слухай швидко, нас викрив якийсь дух і тепер ця будівля спробує знищити
нас. Ми знаємо що тобі її не вистарчає, і заважає вашій зустрічі виключно дана
в’язниця. У тебе є шанс позбутися її вічних страждань, тільки допоможи нам. У
певний час сюди прийде ув’язнений. Допоможи йому втекти! Прикрий його в цьому
місці, і він випустить тебе. Поможи тому, що приносить демонів минулого і
породжує парадокси майбутнього. Проклади для нього дорогу і ...”.

Раптом дівчинка голосно скрикнула. Ведмедик впав на підлогу і зайнявся вогнем
що породжував жахливий смердючий чад, що поглинув дівча. Чадний сморід
заповнив увесь коридор, проникнув у кожну щілину і навіть у  структуру духа в
дротах, заставивши того скрикнути з жаху і скулитися у вибоїні. Коли дим
пропав так само раптово як і з’явився, там де знаходилася дивна дівчинка із
своєю цяцькою зараз знаходилась лише купка попелу. Невідомо звідки повіяв
гидко-зелений вітер, роздуваючи останні докази відбувшогося. Налякана фігура в
проломі стіни озирнулася по бокам, і шумно зітхнувши перетворилася у блакитну
електричну дугу швидко зникнувши у проводці. Коротко блимнули ввімкнувшись
лампи. Коридор знову належав людям.

\section*{} 
 
Навіщо вона дражнила мене? Невже не здогадувалася про мої почуття? Я попав
сюди на короткий термін – лише два роки, але мене підвели власні
почуття... Вона дзвонила, приходила сюди, розмовляла зі мною, а у моїй душі
кріпився і розростався страх за неї, страх перед тим, що я не зможу її
захистити. Страх втратити її на завжди. І ця камера, вона ніби нашіптувала
мені про втрату, про довічну розлуку з коханою. Невидимий шепіт переслідував
мене повсюди: в їдальні, на прогулянці, в сортирі, в потоці душової води і в
погляді кожного наглядача. Він повільно зводив мене з глузду, заставляючи по
ночам бачити страшні картини сповнені зла та насилля. І коли мила подзвонила
мені востаннє, я знав що зроблю... Скло розбилося на тисячі уламків... Темна
гаряча кров фонтаном бризнула із розбитої голови і я побачив свої руки по
лікоть у крові людини, котру обіцяв захищати до кінця віків. Я відчував як
підлога тікає з-під моїх ніг. Я бачив як охорона біжить до мене із витягнутою
зброєю, чув як верещать відвідувачі і проклинають мене в’язні. Але перед очами
у мене стояв образ вбитої дружини і у вухах стояв подавляючий усе маніакальний
глухий, немов кам’яний, сміх.

\section*{}
 
 
Весь персонал, що зібрався у кімнаті страти з нетерпінням очікував прибуття
засудженого. Це не означало, що його особливо ненавиділи (хоча той і вбив свою
дружину ударом в голову на очах двох із них, а ще двоє витирали її смердючі
мізки зі стін та підлоги), просто дні зараз були короткі, а справ було так
багато. Лише Гермес, виконавець вироку, дивився на це як на ще один етап своєї
роботи. Іще один день, ще один мертвий в’язень. Він любив свою роботу і не
думав про своїх підопічних. Відчинилися добре змащені двері, і до “електричної
кімнати” увійшов приречений. Увійшов тихо, із похиленою головою, не дивлячись
по сторонам і думаючи лише про мертву дружину і той дивний шепіт, котрий не
затихав ні на секунду. Так само тихо він сів на “стілець”. Смертник не
поворухнувся поки вправний охоронець прив’язував його і під’єднував усі
контакти. На довгу хвилину у залі стояла незвична і неприродна для цього місця
тиша. Потім тихо загарчав струм і по блоку розлетівся породжений агонією зойк,
від якого стигла кров у жилах. Людина на стільці кричала і виривалася, горячи
на очах. Обличчя і руки покривалися опіками і по приміщенню почав розходитися
характерний запах  паленої плоті. Усі крім ката відвернулися, не в стані
витримати покарання, котре чомусь перетворилося на криваву показову екзекуцію.

Поки в’язня смажили в буквальному значенні цього слова стара будівля
військового форту, яку зараз використовували для утримання злочинців,
здавалося отримувала справжню насолоду від того, що відбувалося у її стінах. У
кожній камері шалено блимало світло, ніби у такт шаленому сміху наркомана,
котрий щойно прийняв дозу. Кожен коридор здавалося був наповнений радістю та
насолодою, хоча в’язні з похмурим виглядом сиділи у камерах.

Раптом усе скінчилося. Видавши останній здавлений агонією хрип, смертник
перетворився на позбавлений будь-яких ознак життя шматок м’яса кольору крові,
що звернулася. Навіть свіжість збагаченого озоном повітря не могла перекрити
того “аромату”, котрий розповсюджувало тіло. “Справу зроблено” сказав
позбавлений емоцій голос ката.

Через кілька секунд у крилі “Б” далеко від камер смертників голуба іскра
дійшла до розподільчого електрощита. Кришка з лязкотом відчинилася і звідти
визирнула фігура як дві краплі води схожа на покійника у блоці “Д”. Примара з
недовірою поглянула на свої скривавлені руки. “Вона не відпускає мене!”. По
в’язниці пройшовся іще один зойк нелюдського страждання, від котрого
здригнулася будівля. А можливо це вона лише посміялася зі своєї нової забавки,
підтверджуючи свою силу і владу над душами...

\part*{Епілог}
\addcontentsline{toc}{part}{Епілог}

\section*{}

У визначений час у щойно звільнену камеру смертників зайшов мовчазний
в’язень. Його засудили до смертної кари. Але він навряд чи усвідомлював це
цілковито. Адже його засудили за вбивство його власної родини, а він навіть не
пам’ятав як це сталося. Винен він чи ні? І взагалі ці дивні провали пам’яті
наповнені злом і кров’ю вже доводили до ручки. Тому поглинутий своїми
роздумами він не помічав нічого навкруги. А вартувало, адже життя мільйонів
людей скоро зміниться власне через нього. А з кутків темряви за ним
спостерігало три пари очей: зацікавлені, жорстокі і підозрілі. І їхні власники
мали свої власні плани щодо в’язня...


 
\end{document}
